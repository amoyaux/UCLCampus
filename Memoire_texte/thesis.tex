\documentclass[11pt, a4paper]{report}

\usepackage{geometry}
\usepackage[utf8]{inputenc}
\usepackage[english]{babel}
\usepackage[T1]{fontenc}

\usepackage{graphicx} % Required for including pictures
\usepackage{float} % Allows putting an [H] in \begin{figure} to specify the exact location of the figure
\usepackage{booktabs}
\usepackage{tabularx}
\restylefloat{table}

\usepackage{wrapfig} % Allows in-line images such as the example fish picture

\usepackage{lmodern} % load a font with all the characters

\linespread{1.2} % Line spacing

%\setlength\parindent{0pt} % Uncomment to remove all indentation from paragraphs

\graphicspath{{Pictures/}} % Specifies the directory where pictures are stored

\renewcommand{\baselinestretch}{1.3}
\headsep = 10.mm
%%%% CHOOSE MARGINS %%%%

\geometry{hmargin=2.5cm,top=3cm,bottom=2.3cm}
%%%% COMMANDS %%%%
%%%%%%%%%%%%%%%%%%%%%%%%%%%%%%%%%%%%% CHANGE HERE %%%%
\renewcommand\title{UCLCampus: a moblie application for UCL students}
\newcommand\nameone{Arnold \textsc{Moyaux}}
\newcommand\nametwo{Baptiste \textsc{Lacasse}}
\newcommand\options{}
\newcommand\supervisor{Yves \textsc{Deville}}
\newcommand\readerone{Kim \textsc{Mens}}
\newcommand\readertwo{hildeberto \textsc{mendonça}}
\newcommand\readerthree{Mathieu \textsc{Zen}}
\newcommand\readerfour{Jorge \textsc{PEREZ MEDINA}}
\newcommand\years{2015-2016}
%%%% BEGIN %%%%
\begin{document}

%%%% COVER %%%%
\thispagestyle{empty}
\noindent\begin{minipage}{.25\textwidth}
\noindent\includegraphics[width=3.6cm]{Images/ucl.jpg}
\end{minipage}
\begin{minipage}{.5\textwidth}
\begin{center}
UNIVERSITE CATHOLIQUE DE LOUVAIN
\\~\\~
ECOLE POLYTECHNIQUE DE LOUVAIN
\\~\\~\\~\\~\\~
\end{center}
\end{minipage}
\begin{minipage}{.25\textwidth}
\hfill\includegraphics[width=3.6cm]{Images/epl.jpg}
\\~\\~\\~\\~
\end{minipage}
\vspace{4.5cm}
\begin{center}
\bfseries{\scshape{\Huge{\title}}}
\end{center}
\vspace{4.5cm}
\begin{minipage}{.5\textwidth}
\begin{tabular}{ll}
Supervisor: & \supervisor
\\ Readers: & \readerone 
\\          & \readertwo 
\\          & \readerthree
\\          & \readerfour
\end{tabular} 
\\~\\~
\end{minipage}
\begin{minipage}{.5\textwidth}
\begin{tabular}{l}
Thesis submitted for the Master's degree
\\ in computer science and engineering
\\ options: \
\\ by : \nameone
\\ 	   \nametwo
\end{tabular}
\end{minipage}
\vfill
\begin{center}
Louvain-la-Neuve
\\ Academic year \years
\end{center}
%%%% END COVER %%%%
%----------------------------------------------------------------------------------------
%	TABLE OF CONTENTS
%----------------------------------------------------------------------------------------

\tableofcontents % Include a table of contents

\newpage % Begins the essay on a new page instead of on the same page as the table of contents 


\chapter{Introduction} % Major section

Brief introduction of the project, the goals and the contents of the rest of the thesis.


\chapter{Background}

In this section, we will look at the different existing technologies relating to the different aspects of our project and will explain the choices we made. // TODO

\section{Cross-platform mobile development tools}

In each of these sections, we will detail the different approaches one could choose to develop a cross-platform moblie applications. We will also present several frameworks using these approaches. We will then compare them and choose one of those approaches for the rest of the project. //TODO
\subsection{The native approach}

The first approach we considered for our project was what we call a native approach. The native approach consists in using the native technology and language for each platform, for instance Java for Android and Objective-C for iOS.  

\begin{table}[H]
\begin{tabularx}{\linewidth}{>{\parskip1ex}X@{\kern4\tabcolsep}>{\parskip1ex}X}
\toprule
\hfil\bfseries Pros
&
\hfil\bfseries Cons
\\\cmidrule(r{3\tabcolsep}){1-1}\cmidrule(l{-\tabcolsep}){2-2}

Best achievable performance\par
Always up-to-date with the latest API\par
Can use any platform\par

&

Low maintainability\par
Harder to find contributors fluent in all technologies\par
Can lead to different versions of the application\par


\\\bottomrule
\end{tabularx}
\caption{Pros and cons of the native approach}
\end{table}

\subsection{The web approach}
A second approach we considered was the web approach. This approach consists in using HTML5 to develop an application that will be usable on any platform.


\begin{table}[H]
\begin{tabularx}{\linewidth}{>{\parskip1ex}X@{\kern4\tabcolsep}>{\parskip1ex}X}
\toprule
\hfil\bfseries Pros
&
\hfil\bfseries Cons
\\\cmidrule(r{3\tabcolsep}){1-1}\cmidrule(l{-\tabcolsep}){2-2}

Can be used on any mobile platform\par
Easy to find contributors fluent in HTML5\par
Easy to maintain\par

&

Doesn't have access to native platform features\par
Harder to implement local storage/security (//TODO NEED BETTER SOURCE)\par
Not as performant as native\par



\\\bottomrule
\end{tabularx}
\caption{Pros and cons of the web approach}
\end{table}
\subsection{The hybrid approach}

The last approach to develop a mobile application is called the hybrid approach. An hybrid app is mostly built using HTML5 and JavaScript and is then wrapped inside a thin native container, giving it access to native features.

\begin{table}[H]
\begin{tabularx}{\linewidth}{>{\parskip1ex}X@{\kern4\tabcolsep}>{\parskip1ex}X}
\toprule
\hfil\bfseries Pros
&
\hfil\bfseries Cons
\\\cmidrule(r{3\tabcolsep}){1-1}\cmidrule(l{-\tabcolsep}){2-2}

Can be used on any mobile platform\par
Easy to find contributors fluent in HTML5 and JavaScript\par
Easy to maintain\par

&

Not as performant as native\par



\\\bottomrule
\end{tabularx}
\caption{Pros and cons of the hybrid approach}
\end{table}

\subsection{Our choice}

\section{Open-source project and code sharing}

In this section, we will explain the choices we made concerning the code sharing platforms we used as well as the licenses we used to protect our work.

\section{Project Management Methodologies}

Here we detail the choices we made as to how we were going to manage de different parts of the project.

\chapter{Functionalities of UCLCampus}

In this part, we will show how we defined the relevant functionalities of our application as well as the user interface.

\section{Choice of functionalities and sections}



In order to define what kind of functionalities we wanted to be part of our application, we needed to know what the students needed. The first step was to define a number of user stories that we would then translate into functionalities.\\ 
We split our user stories into several categories:

\begin{itemize}

\item Studies: anything directly related to a user's studies, for instance his classes, the lecture halls or the libraries.
\item Campus: anything related to student life in the campus but not related to the user's studies. For instance "Kot à Projets" or "Cercles".
\item City: anything related to the city the user is in but not related to the university. For instance a cinema or restaurants.
\item Tools: the tools offered by the application that might relate to several other categories. For example the map.
\item Settings: the settings of the application. For example the language or the currently selected campus.

\end{itemize}

We also define two types of users:

\begin{itemize}

\item Students: students can access all the functionalities of the application using their UCL login information. Indeed, some functionalities are student specific. For instance, it wouldn't make sense for a person who isn't a student to try to access his or her schedule.

\item Users: users are people who aren't students but might still be interested in some functionalities the application has to offer. 

\end{itemize}

In our user stories, any story starting by 'as a student' cannot be used by users while any story starting by 'as a user' can be used by both users and students.\\

We will now give a list of the different user stories we thought of for each of our categories.

\subsubsection{Studies}

\begin{itemize}

\item Schedule
\begin{itemize}
\item As a student, I can access my schedule in order to know when my courses are given.
\item As a student, I want to know where a course is given.
\item As a student, I want to know the name of a teacher giving a certain course of my schedule.
\item As a student, I can export my schedule to my phone's agenda so that I don't need Internet access to see it.
\end{itemize}

\item Libraries
\begin{itemize}
\item As a user, I can see whether a library is open or closed.
\item As a user, I can display the address of any library.
\item As a user, I can have a GPS guide to access libraries from my location.
\end{itemize}

\item Lecture halls
\begin{itemize}
\item As a user, I can check the address of any lecture hall.
\item As a user, I can have a GPS guide to access lecture halls from my location.
\end{itemize}

\item Websites
\begin{itemize}
\item As a user, I can quickly access the moodle website through the application.
\item As a user, I can quickly access the UCL website through the application.
\end{itemize}

\end{itemize}

\subsubsection{Campus}

\begin{itemize}

\item{Events}
\begin{itemize} 
\item As a user, I can see a list of events taking place in my campus.
\item As a user, I can sort the events by category.
\end{itemize}

\item{Kots à Projet}
\begin{itemize}
\item As a user, I can check Kots à Projet to know their address and projects.
\item As a user, I can have a GPS guide to access Kots à Projet from my location.
\end{itemize}

\item{Cercles}
\begin{itemize}
\item As a user, I can check "Cercles" to know their address.
\item As a user, I can have a GPS guide to access "Cercles" from my location.
\end{itemize}

\item{Restaurants Universitaires}
\begin{itemize}
\item As a user, I can see the different "Restaurants Universitaires" in my campus.
\item As a user, I can check the menu of the "Restaurants Universitaires".
\item As a user, I can have a GPS guide to access "Restaurant Universitaires" from my location.
\end{itemize}

\item{Sports}
\begin{itemize}
\item As a user, I can see a list of sports organized in my campus.
\item As a user, I can sort the sports by day or by sport.
\end{itemize}

\end{itemize}

\subsubsection{City}

\begin{itemize}

\item Tourism
\begin{itemize} 
\item As a user, I can see the address of the city's information center.
\item As a user, I can see a list of the museums of the city I'm in.
\item As a user, I can see whether a museum is opened or closed.
\end{itemize}

\item Activities
\begin{itemize} 
\item As a user, I can see the address of the city's cinema in order to access it with the help of a GPS guide.
\item As a user, I can see several activities I can do in the city I'm in.

\end{itemize}

\item Restaurants and bars
\begin{itemize} 
\item As a user, I can see a list of the restaurants of the city I'm in.
\item As a user, I can see a list of the bars of the city I'm in.
\end{itemize}

\end{itemize}

\subsubsection{Tools}

\begin{itemize}
\item As a user, I can access a map of the city I'm in in order to check points of interests.
\item As a user, I can receive help from a GPS guide in order to access a location of my choice on the map.
\end{itemize}

\subsubsection{Settings}

\begin{itemize}
\item As a user, I can change the application's language to French, English or Dutch.
\item As a user, I can select my campus.
\end{itemize}

\section{User interface}

\chapter{Implementation}

Here we will explain the overall architecture of the application. We will also explain some aspects we considered when implementing the application.

\section{Architecture}

\section{Coding standards}

\section{Security}

\section{Information retrieval}
\newpage

\chapter{The application}

In this section we will present the application as we implemented it.

\section{The application UCLCampus}

\subsection{Studies}

\subsection{Campus}

\subsection{City}

\subsection{Tools}

\subsection{Others}


\section{Modularity and how to add a new functionality}

\section{Future functionalities and possible improvements}

\chapter{Analysis}

Here we will reflect about the many choices we made and try to decide wether they were the right ones or not.

\section{Ionic framework}

\section{GitHub}

\section{Project Management}


\chapter{Conclusion} 

\chapter{Bibliography}



\end{document}