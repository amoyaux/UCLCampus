

\documentclass[12pt]{article} % Default font size is 12pt, it can be changed here

\usepackage{geometry} % Required to change the page size to A4
\geometry{a4paper} % Set the page size to be A4 as opposed to the default US Letter

\usepackage{graphicx} % Required for including pictures

\usepackage{booktabs}
\usepackage{tabularx}
\usepackage{float}
\restylefloat{table}

\usepackage{float} % Allows putting an [H] in \begin{figure} to specify the exact location of the figure
\usepackage{wrapfig} % Allows in-line images such as the example fish picture

\usepackage[utf8]{inputenc}
\usepackage[T1]{fontenc}
\usepackage{lmodern} % load a font with all the characters

\linespread{1.2} % Line spacing

%\setlength\parindent{0pt} % Uncomment to remove all indentation from paragraphs

\graphicspath{{Pictures/}} % Specifies the directory where pictures are stored

\begin{document}

%----------------------------------------------------------------------------------------
%	TITLE PAGE
%----------------------------------------------------------------------------------------

\begin{titlepage}

\newcommand{\HRule}{\rule{\linewidth}{0.5mm}} % Defines a new command for the horizontal lines, change thickness here

\center % Center everything on the page


\textsc{\LARGE Université Catholique de Louvain}\\[1.5cm] %
\vspace{3cm}
\textsc{\Large UCLCampus: a moblie application for UCL students}\\[0.5cm]
\textsc{\large Master's Thesis}\\[0.5cm] 
\vspace{3cm}

\textsc{\large\emph{Authors:} }\\[0.5cm] 
\textsc{\large Arnold \textsc{Moyaux}}\\[0.5cm] 
\textsc{\large Baptiste \textsc{Lacasse}}\\[0.5cm] 
\vspace{1cm}
\textsc{\large\emph{Supervisor:} }\\[0.5cm]
\textsc{\large Pr. Yves \textsc{Devilles}}\\[0.5cm]

\vspace{2cm}
{\large \today}\\[3cm] % Date, change the \today to a set date if you want to be precise


\vfill 

\end{titlepage}

%----------------------------------------------------------------------------------------
%	TABLE OF CONTENTS
%----------------------------------------------------------------------------------------

\tableofcontents % Include a table of contents

\newpage % Begins the essay on a new page instead of on the same page as the table of contents 


\section{Introduction} % Major section

Brief introduction of the project, the goals and the contents of the rest of the thesis.


\section{Preliminary analysis}

In this section, we will look at the different existing technologies relating to the different aspects of our project and will explain the choices we made. // TODO

\subsection{Cross-platform mobile development tools}

In each of these sections, we will detail the different approaches one could choose to develop a cross-platform moblie applications. We will also present several frameworks using these approaches. We will then compare them and choose one of those approaches for the rest of the project. //TODO
\subsubsection{The native approach}

The first approach we considered for our project was what we call a native approach. The native approach consists in using the native technology and language for each platform, for instance Java for Android and Objective-C for iOS.  

\begin{table}[H]
\begin{tabularx}{\linewidth}{>{\parskip1ex}X@{\kern4\tabcolsep}>{\parskip1ex}X}
\toprule
\hfil\bfseries Pros
&
\hfil\bfseries Cons
\\\cmidrule(r{3\tabcolsep}){1-1}\cmidrule(l{-\tabcolsep}){2-2}

Best achievable performance\par
Always up-to-date with the latest API\par
Can use any platform\par

&

Low maintainability\par
Harder to find contributors fluent in all technologies\par
Can lead to different versions of the application\par


\\\bottomrule
\end{tabularx}
\caption{Pros and cons of the native approach}
\end{table}

\subsubsection{The web approach}
A second approach we considered was the web approach. This approach consists in using HTML5 to develop an application that will be usable on any platform.


\begin{table}[H]
\begin{tabularx}{\linewidth}{>{\parskip1ex}X@{\kern4\tabcolsep}>{\parskip1ex}X}
\toprule
\hfil\bfseries Pros
&
\hfil\bfseries Cons
\\\cmidrule(r{3\tabcolsep}){1-1}\cmidrule(l{-\tabcolsep}){2-2}

Can be used on any mobile platform\par
Easy to find contributors fluent in HTML5\par
Easy to maintain\par

&

Doesn't have access to native platform features\par
Harder to implement local storage/security (//TODO NEED BETTER SOURCE)\par
Not as performant as native\par



\\\bottomrule
\end{tabularx}
\caption{Pros and cons of the web approach}
\end{table}
\subsubsection{The hybrid approach}

The last approach to develop a mobile application is called the hybrid approach. An hybrid app is mostly built using HTML5 and JavaScript and is then wrapped inside a thin native container, giving it access to native features.

\begin{table}[H]
\begin{tabularx}{\linewidth}{>{\parskip1ex}X@{\kern4\tabcolsep}>{\parskip1ex}X}
\toprule
\hfil\bfseries Pros
&
\hfil\bfseries Cons
\\\cmidrule(r{3\tabcolsep}){1-1}\cmidrule(l{-\tabcolsep}){2-2}

Can be used on any mobile platform\par
Easy to find contributors fluent in HTML5 and JavaScript\par
Easy to maintain\par

&

Not as performant as native\par



\\\bottomrule
\end{tabularx}
\caption{Pros and cons of the hybrid approach}
\end{table}

\subsubsection{Our choice}

\subsection{Open-source project and code sharing}

In this section, we will explain the choices we made concerning the code sharing platforms we used as well as the licenses we used to protect our work.

\subsection{Project Management Methodologies}

Here we detail the choices we made as to how we were going to manage de different parts of the project.

\section{Project definition}

In this part, we will show how we defined the relevant functionalities of our application as well as the user interface.

\subsection{Choice of functionalities and sections}

\subsection{User interface}

\section{Project implementation}

Here we will explain the overall architecture of the application. We will also explain some aspects we considered when implementing the application.

\subsection{Architecture}

\subsection{Coding standards}

\subsection{Security}

\subsection{Information retrieval}
\newpage

\section{The application}

In this section we will present the application as we implemented it.

\subsection{Functionalities}

\subsubsection{Studies}

\subsubsection{Campus}

\subsubsection{City}

\subsubsection{Tools}

\subsubsection{Others}


\subsection{Modularity and how to add a new functionality}

\subsection{Future functionalities and possible improvements}

\section{Post-implementation analysis}

Here we will reflect about the many choices we made and try to decide wether they were the right ones or not.

\subsection{Ionic framework}

\subsection{GitHub}

\subsection{Project Management}


\section{Conclusion} 

\section{Bibliography}



\end{document}